\chapter{Review of Functions}
\section{Introduction}


\section{Linear Functions}
A linear function is a function that can be written in the form $f(x) = mx + b$, where $m$ is the slope of the line and $b$ is the $y$-intercept.

\subsection{Hyperbolic Functions}

Hyperbolic cosine 
\begin{equation}
	\cosh x = \frac{e^x + e^{-x}}{2}
\end{equation}

\chapter{Limits}
\section{Introduction}
This is the introduction section of my document.
\section{Intutive Definition of a Limit}
Imagine you’re standing on the shore, looking out at the ocean. You see a boat far away, and it’s moving towards you. As the boat gets closer, it appears larger and clearer, but if it were to keep coming closer indefinitely, it would eventually reach you, right at your feet. Now, we might say that the boat “approaches” you as it moves closer and closer.

In mathematics, a limit is a way of describing what happens when we look at how something changes as we move closer to a certain point. Think of it as focusing on what happens in the “long run” as we approach a specific point, rather than what happens exactly at that point.

Let’s use an example with numbers: Imagine you have a sequence of numbers that gets closer and closer to 10, like 9, 9.9, 9.99, 9.999, and so on. Even though none of these numbers are exactly 10, we can say that “in the limit,” these numbers are approaching 10. The limit of this sequence is 10 because, as we go further along in the sequence, the numbers get arbitrarily close to 10.

In this sense, a limit is about getting closer and closer to something without necessarily ever reaching it. It’s about the behavior of a function or a sequence as we move toward a certain point or as the input grows indefinitely.

In mathematical terms, if we say the limit of \(f(x)\) as \(x\) approaches \(a\) is \(L\), we mean that we can get \(f(x)\) as close as we want to \(L\) by taking x sufficiently close to \(a\), but not necessarily equal to \(a\).

\section{Preview of Calculus}


\begin{formula}
    {Formula}
	\begin{equation} 
		m_{sec}=\frac{f(x)-f(a)}{x-a}
	\end{equation}
\end{formula}



\section{The Limit of A Function}
\section{The Limit Laws}
\section{Continuity}
\section{The Precise Definition of a Limit}

\begin{formula}
    {Precise Definition of a Limit}
	\begin{equation} 
		\lim_{x \to a} f(x) = L
	\end{equation}
\end{formula}



\begin{definition}
    {Definition of a Limit}
    The limit of a function $f(x)$ as $x$ approaches $a$ is $L$ if for every $\epsilon > 0$ there exists a $\delta > 0$ such that if $0 < |x - a| < \delta$, then $|f(x) - L| < \epsilon$.
\end{definition}

\begin{example}
    {Example 1}
  Enter an example here. 
\end{example}

\begin{important} 
    {Important} It is important that... 
\end{important}

\begin{formula}
    {Formula}
    Enter a formula here.
\end{formula}

\begin{proof}
    {Proof} This is a proof 
\end{proof}

\begin{equation} 
    s(t)= \text{position of the object at time $t$}
\end{equation}

\begin{example} 
{Example 2.2}
\[s(t)=16t^2+64\]
a) \([0.49, 0.50]\) \\
\begin{equation}
    \frac{s(0.5)-s(0.49)}{0.5-0.49}=-15.84
\end{equation}
b) \([0.50, 0.51]\) \\
\begin{equation}
    \frac{s(0.51)-s(0.5)}{0.51-0.5}=16.16u
\end{equation}

\end{example}